\documentclass[utf8]{gradu3}
% Jos työ on kandidaatintutkielma eikä pro gradu, käytä ylläolevan asemesta
%\documentclass[utf8,bachelor]{gradu3}
% Jos kirjoitat englanniksi, käytä ylläolevan asemesta
%\documentclass[utf8,english]{gradu3}
% tai
%\documentclass[utf8,bachelor,english]{gradu3}

\usepackage{graphicx} % kuvien mukaan ottamista varten

\usepackage{amsmath} % hyödyllinen jos tekstisi sisältää matikkaa,
                     % ei pakollinen

\usepackage{booktabs} % hyvä kauniiden taulukoiden tekemiseen
\usepackage{textgreek} % tarvitaan kreikkalaisille kirjaimille
\usepackage{comment} % paketti suurille kommenttilohkoille

% HUOM! Tämän tulee olla viimeinen \usepackage koko dokumentissa!
\usepackage[bookmarksopen,bookmarksnumbered,linktocpage]{hyperref}

\addbibresource{malliopas.bib} % Lähdetietokannan tiedostonimi

\begin{document}

\title{Kirjallisuuskartoitus laadullisten simulaatioiden käyttöön}
\translatedtitle{A literature review about using simulations in qualitative research}
\studyline{Tietotekniikka}
\avainsanat{%
  Laadullinen tutkimus,
  Simulaatio,
  Laadullinen simulaatio,
  Systemaattinen kirjallisuuskartoitus
  }
\keywords{Qualitative research, Simulation, Qualitative simulation, Systematic mapping study}
\tiivistelma{%
  TODO
}
\abstract{%
  
}

\author{Jaakko Palm}
\contactinformation{\texttt{jaakko.j.palm@student.jyu.fi}}
% jos useita tekijöitä, anna useampi \author-komento
\supervisor{Timo Tiihonen}
% jos useita ohjaajia, anna useampi \supervisor-komento

\maketitle

\mainmatter

\chapter{Johdanto} \label{johdanto}
Usein tieteellisiä tutkimuksia puhutessa viitataan joko empiirisiin tai teoreettisiin tutkimuksiin. Empiirisessä tutkimuksessa tehdään ja kerätään havaintoja sekä analysoidaan ja mitataan tutkimuksen kohteena olevaa tapahtumaa tai järjestelmää. Teoreettisessa tutkimuksessa pyritään hahmottamaan malleja ja rakenteita aiemman tutkimuskirjallisuuden pohjalta. Tutkimusmenetelmät empiirisille tutkimuksille voivat olla tyypiltään \textit{laadullisia (eng. qualitative)} tai \textit{määrällisiä (eng. quantitative)}.

Yksi tunnettu menetelmä empiirisille tutkimuksille on tietokonesimulaatio. Tietokonesimulaatio on ohjelma, jolla pyritään mallintamaan reaalimaailman käyttäytymistä. 
Simulaatiot perustuvat luotuihin malleihin, jotka pyrkivät kuvaamaan halutun kohteen käyttäytymistä. Mallin luomista kutsutaan mallintamiseksi, joka on erittäin tärkeä prosessi simulointia, sillä se määrittää mitä ja miten järjestelmää simuloidaan.  

Simulaatioilla kyetään keräämään dataa ja havaintoja tutkittavasta aiheesta suhteellisen helposti ja halvalla, mikä tekee siitä oleellisen tutkimusmenetelmän tieteellisiin tutkimuksiin. Simulaatiot perustuvat malleihin, jotka pyrkivät mallintamaan kohteeseen tai järjestelmään tyypillisiä ominaisuuksia. Yksinkertaisimmillaan simulaatio on ohjelma, joka mallintaa tapahtumaa. Ohjelmalle syötetään tapahtuman muuttujat ja ohjelma antaa lopputuloksen siitä, mitä simulaation mallin mukaan tulisi tapahtumaan. 

Simulaatioiden avulla tutkijat voivat suorittaa kokeita, jotka voivat olla epäkäytännöllisiä tai mahdottomia todellisessa maailmassa, säilyttäen samalla laajan tilastollisen analyysin. Koska määrällisessä tutkimuksessa pyritään selittämään ilmiöitä perustuen dataan ja datan väliseen suhteeseen, niin simulaatiot ja simulointi on erittäin oleellisia työkaluja määrällisille tutkimuksille. 

Simulaatioiden käyttö laadullisissa tutkimuksissa on monimutkaisempi asia, jonka mahdollisuus taas riippuu laadullisen tutkimuksen tyypistä ja kohteesta. Simulaatioiden käyttö laadullisissa tutkimuksissa on kuitenkin laaja alue, esimerkiksi \textcite{eldabi2002quantitative} ehdottivat tutkimuksessaan miten simulaatioilla voidaan välttää laadullisten menetelmien tyypillisiä ongelmia, kuten työlästä datan keruuta. \textcite{kuipers1986qualitative} esittivät termin \textit{laadullinen simulaatio (eng. qualitative simulation)}, mikä viittaa simulaatioihin, jotka eivät tarvitse tarkkoja numeerisia parametreja toimiakseen, mikä tekee niistä oleellisen laadullisille tutkimuksille.

Laadulliset simulaatiot ovat hyödyllisiä monessa eri tilanteessa. Jos tutkimuskohteen tiedossa on aukkoja tai ei muuten pystytä antamaan tarkkoja parametrejä simulaatiolle, eli ei voida toteuttaa normaalia määrällistä simulaatiota. Näissä tapauksissa laadullisen simulaation toteuttaminen voi olla kuitenkin mahdollista. Usein myös laajoissa kompleksisissa kohteissa on erittäin vaikeaa tehdä normaalia simulaatiota, joten laadullinen simulaatio voi olla järkevämpi vaihtoehto.

Laadullisille simulaatioille on myös toteutettu työkaluja, kuten Garp3-ohjelmisto \parencite{bredeweg2007garp3} ja QSIM-algoritmi \parencite{helgstrand2004qsim} ovat tarkoitettu laadullisten simulaatioiden toteutettamiseen. Lisäksi on myös useita laadullisia tutkimuksia, joissa on käytetty laadullisia simulaatioita. Esimerkiksi \textcite{cao2019depth} tekivät tutkimuksen kaivosmiesten turvallisuuskäyttäytymisestä käyttäen simulaatiota ja tämän perusteella päätellen, miten ongelmatilanteissa tulisi reagoida. 

Tämän tutkimuksen tavoitteena on toteuttaa systemaattinen kirjallisuuskartoitus laadullisten simulaatioiden käytöstä ja kerätyn aineiston perusteella vastata seuraaviin kysymyksiin:
\begin{itemize}
    \item Minkä alan tutkimuksissa on eniten käytetty laadullisia simulaatioita?
    \item Mitä eri työkaluja löytyy laadullisten simulaatioiden luomiseen?
    \item Millaisia tuloksia laadullisilla simulaatioilla pyritään hakemaan?
\end{itemize}

Luvussa 2 kerrotaan simulaatioista ja laadullisista simulaatioista yleisesti. Luvussa 3 käydään läpi tutkimuksen asetelma ja käytetyt menetelmät. Luvussa 4 esitellään kirjallisuuskartoituksen tulokset. Luvussa 5 pohditaan tulosten merkitystä ja luvussa 6 on tutkimuksen yhteenveto.

\chapter{Mallintaminen ja simulaatiot}
Tässä kappaleessa käydään läpi gradulle oleellisten asioiden taustaa. Luvussa \ref{mallintaminen} kerrotaan mallintamisesta ja miksi se on oleellista simuloinnin näkökulmasta. Luku \ref{simulaatio} kertoo lyhyesti simulaation historiasta ja kehityksestä. Luku \ref{laadullinen simulaatio} kertoo mikä on laadullinen simulaatio ja sen historiasta.

\section{Mallintaminen} \label{mallintaminen}
Mallintaminen on prosessi, jolla pyritään tuottamaan malli. Malli on jäljitelmä halutun järjestelmän toimintannasta. \parencite{maria1997introduction}. Oleellisen mallin pitää olla lähellä todellista järjestelmää ja sisältää suurimman osan sen tärkeimmistä ominaisuuksista, mutta mallin ei pidä olla myöskään liian monimutkainen, jotta sen ymmärtäminen ja testaaminen olisi mahdollista \parencite{maria1997introduction}. 
\textcite{maria1997introduction} mukaan mallintaminen on tärkein osuus simulaatiotutkimuksessa. Simulaatiota varten luotua mallia kutsutaan simulaatiomalliksi.

Simulaatiomalleja on useita eri tyyppejä. Kuten determistinen, jonka syöte- ja tulosmuuttujat ovat kiinteitä arvoja. Stokastinen, jossa ainakin yksi syöte- tai tulosmuuttuja on satunnainen. Staattinen, jossa ajankulua ei oteta huomioon tai dynaaminen, jossa ajankulu vaikuttaa muuttujiin. Mallin usein ovat stakastisia ja dynaamisia \parencite{maria1997introduction}. Simulaatiot usein perustuvat simulaatiomalleihin. Simulaatiomalli sisältää yleensä parametreja, jotka mahdollistavat
mallin uudelleenkäyttämisen \parencite{introduction2005simulation}. 

\textcite{maria1997introduction} mukaan Simulaatiomallit koostuvat usein seuraavista komponenteista: järjestelmäkokonaisuudet, syöttemuuttujat, suorituskyvyn mittaajat,
ja osien toiminnalliset suhteet. Mallintamisprosessi tiivistetysti etenee askeleittain, alkaen ongelman tunnistamisella ja muotoilulla, jonka perusteella toteutetaan datan keruu, josta malli muotoillaan, kehitetään ja validoitaan, lopuksi mallin toiminta ja logiikka dokumentoidaan myöhempää käyttöä varten.

\section{Simulaatio} \label{simulaatio}
Simulaatio viittaa reaalimaailman prosessin tai järjestelmän jäljittelyyn \parencite{banks1999introduction}. Käsitteenä simulaatio on ollut jo useampaa sataa vuotta, yksi varhaisimmista tiedetyistä simulaatio metodeista on Monte Carlo-metodi, joka sai alkunsa vuonna 1777 \parencite{HistoryOfSimulation}. 1940-luvun puolivälissä simulaatiot saivat kaksi suurta kehitysaskelta, ensimmäisten tietokoneiden rakentaminen ja Monte Carlo-metodin ensimmäiset toteutukset tietokoneen avulla \parencite{HistoryOfSimulation}. 

\cite{banks1999introduction} mukaan simulaatiolla on useampi eri hyöty, mitä tulee sen käytössä tutkimusten työkaluna. Simulaatioilla voidaan säästää resursseja, sillä ei tarvitse hankkia materiaaleja erikseen testaamista varten. Simulaatioilla voidaan nopeuttaa sen jäljittelemiä reaalimaailman toimintoja, mikä säästää aikaa jos tutkittava tapahtuma vie paljon aikaa reaalimaailmassa. Simulaatioilla voidaan tutkia tapahtumien syitä tarkemmin, koska simulaation tekijällä on hallitsemus mitä simulaatiossa tapahtuu. Simulaatioilla voidaan helposti analysoida miten eri muuttuvat tekijät voivat vaikuttaa tapahtumiin. 


\section{Laadullinen simulaatio} \label{laadullinen simulaatio}
Laadullinen simulaatio viittaa simulaatioon, joka on keskeinen päättelyprosessi laadullisessa kausaalisessa päättelyssä \parencite{kuipers1986qualitative}. Laadullinen simulointi on mallinnustapa, joka korostaa järjestelmän käyttäytymisen laadullisia näkökohtia tarkkojen numeeristen arvojen sijaan. Usein tämän kaltaiset simulaatiot määritellään osana \textit{laadullisen päättelyn (eng. qualitative reasoning)} tieteenalaa
%
\parencites%
  {parallelQualitativeSimulation1997}%
  {kuipers1986qualitative}%
\relax.
%
Laadullinen päättely viittaa tieteellisiin tapoihin joilla voidaan tutkia kohdetta, jonka toiminnan tiedosta on puutteita tai aukkoja.

Laadullisella simulaatiolla voidaan tutkia tietyn järjestelmän dynamiikkaa ilman tarkkoja parametriarvoja \parencite{cosme2023history}. On myös mahdollista toteuttaa simulaatio, joka on ominaisuuksiltaan sekä laadullinen että määrällinen. Näistä simulaatioista käytetään termiä \textit{ puoli-määrällinen simulaatio (eng.semi-quantitative simulation)}\parencite{semiHybrid1997qualitative}.

Laadullisten simulaatioiden luonttin löytyy useita eri metodeja ja työkaluja.
Esimerkiksi Qsim on algoritmi, jolla voidaan luoda laadullisia malleja simulaatioille \parencite{kuipers1986qualitative}. Lisäksi on myös \textit{sumeaan teoriaan (eng. fuzzy set theory)} perustuva \textit{sumea laadullinen simulaatioalgoritmi (eng.fuzzy qualitative simulation algorithm)} auttaa myös toteuttamaan laadullisia malleja \parencite{shen1993fuzzy}.


\subsection{Hyödyt}
Laadullisella simuloinnilla on tietysti samat yleiset hyödyt kuin muilla simulaatiolla, mainittuna luvussa \ref{simulaatio}. Laadulliset simulaatiot toimivat hyvin tapauksissa, joissa simuloitavan kohteen tai järjestelmän toiminnasta ei ole täydellistä tietoa tai tarvittava tieto on muuten aukollinen \parencite{kuipers1986qualitative}. 

Simulaatiomallin on mahdollista toteuttaa sekä laadullisia että määrällisiä ominaisuuksia samaan aikaan. Laadullisen simulaation ja määrällisen simulaatioiden yhdistinen voi luoda simulaatioita, jotka osoittavat molempien tapojen vahvuuksia \parencite{semiHybrid1997qualitative}. Yksi mahdollinen tapa puolilaadulliselle simulaatiolle on yksinkertaisimmillaan määrällinen simulaatio, jossa käytetään numeerisia välejä ominaisuuksissa, joista ei ole täydellistä määrällistä tietoa \parencite{semiHybrid1997qualitative}.

\subsection{Ongelmat}
\textcite{QualSimTheoryApplications2013} mukaan laadullinen simulaatio on hitaampi tapauksissa joissa voidaan toteuttaa normaali määrällinen simulaatio. Eli laadullinen simulaatio on kompleksisuutensa mukaan yleensä laskennalliseti hitaampi kuin määrällinen. 
\textcite{QualSimTheoryApplications2013} mukaan myös kun parametrien määrä kasvaa laadullisissa simulaatioissa, niiden simulaationopeus laskee merkittävästi. 
Näitä ongelmia voidaan mahdollisesti rajata käyttämällä hybridimalleja, mutta 
tapauksissa joissa täytyy toteuttaa täysin laadullinen simulaatio, niin koneelliset resurssit voivat mahdollisesti olla ongelmana. 

\textcite{soundQualSimImpossible} mukaan täydellinen laadullinen simulaatio on mahdotonta.
Tutkimus keskittyy suurimmin osin QSIM-algoritmin kritiikkiin, mutta \textcite{soundQualSimImpossible} esittivät myös yleisempiä ongelmia laadulliseen simulaatioon, kuten tutkijoilta puuttuu usein täydellinen tieto muotoillakseen tarkkoja numeerisia yhtälöitä ja laadullisessa päättelyssä voidaan käyttää heikkoja esityksiä epätäydellisten järjestelmien mallintamiseen.

 \section{Tutkielman tarkoitus}
Vaikka laadullinen mallinnus ja simulointi on laaja alue, josta löytyy tutkimuksia useammalta eri tieteenalalta, aiheesta ei ole löytynyt aikaisempaa yleiskatsausta. Aiheeseesta on tehty tutkimuksia esimerkiksi eri laadullisten simulaatioiden tarkkuudesta \parencite{FisherManagmentTechniques2024} ja eri laadullisten simulaatioiden menetelmien toimivuudesta \parencite{qualitativeSimTechniquesAssesment1992}, mutta tämän tutkielman kirjoitus ajankohdalla ei löytynyt laajempaa yleiskatsausta mitä tulee laadullisten simulaatioiden levinneisyyteen, työkaluihin ja käyttötarkoituksiin.  

Gradun tavoitteena on saada kyseinen yleiskatsaus laadullisista simulaatioista systemaattisen kirjallisuuskartoitusta käyttäen, jotta saataisiin haluttu yleiskäsitys millä aloilla laadulliset simulaatioit ovat levinneet eniten, miten paljon laadullisten simulaatioiden luontiin ja mallintamiseen löytyy työkaluja ja ohjeita sekä minkäkaltaisia tuloksia laadullisilla simulaatioilla etsitään.


\chapter{Tutkimusasetelma}
Tässä kappaleessa käydään läpi valittu tutkimusasetelma ja sen vaiheet. Luvussa \ref{tutkimusmenetelmä} kerrotaan valitusta tutkimusmenetelmästä ja miksi se on valittu. Luvussa \ref{hakukoneiden valinta} käydään läpi tutkimuksen hakukohteissa käytetyt avaintermit ja hakujen rajaukset. Luvussa \ref{valintakriteerit} kerrotaan tutkimuksessa käytetyt kriteerit tutkimusaineistoon sopivista artikkeleista. Luku \ref{valintaprosessi} kertoo aineiston keruun prosessista ja \ref{aineiston analysointi} kuvailee miten kerättyä aineisto analysoitiin.

\section{Tutkimusmenetelmä} \label{tutkimusmenetelmä}
Tutkielman menetelmäksi valittiin kirjallisuuskartoitus, koska halusin saada yleiskuvan laadullisten simulaatioiden käytöstä tieteellisenä tutkimusmenetelmänä. \textcite{keele2007guidelines} mukaan systemaattisella kirjallisuuskartoituksella voidaan saadaan laaja yleiskatsaus tutkimuskohteesta, mikä tekee siitä oleellisen menetelmän tutkimukselle. Systemaattinen kirjallisuuskartoitus antaa halutun yleiskuvan laadullisista simulaatioista ja menetelmä sopii hyvin vastaamaan luvussa \ref{johdanto} esitettyihin tutkimuskysymyksiin.


\section{Hakukoneiden valinta} \label{hakukoneiden valinta}
Laadulliset simulaatiot aiheena oli odotettua laajempi, joten gradun resurssien takia haku jouduttiin rajaamaan yhteen tutkimustietokantaa. Hakukriteereitä rajatessa, tutkimuksessa testattiin valitun hakukohteen ScienceDirectin lisäksi IEEE Xplore-tutkimustietokantaa \footnote{\url{https://ieeexplore.ieee.org/}} ja Google Scholarin sekä Jyväskylän yliopiston (JYKDOK) \footnote{\url{https://jyu.finna.fi/}} hakukoneita. Näistä IEEE Xplore ja JYKDOK jäivät liian vähäisiksi. Google Scholarin hauissa taas saatujen artikkelien määrä oli liian suuri, mitä hankaloitsi Scholarin vajaat hakujen rajausvaihtoehdot. Testatut hakukoneet ja niiden tuloset on esillä taulukossa \ref{table: hakutulokset}.

Tutkimukset aineistoon kerättiin ScienceDirect-tutkimustietokannasta \footnote{\url{https://www.sciencedirect.com/}}. Hakuterminä käytettiin "qualitative simulation" ja haun rajaamisessa artikkelit rajattiin englannin kielisiin tutkimusartikkeleihin vuosina 2000-2024. 

\begin{table}[]
\centering
\begin{tabular}{|l|l|}
\hline
\textbf{Hakukohde} & \textbf{Hakutulokset} \\ \hline
Google Scholar     & 11300                 \\ \hline
JYKDOK             & 13                    \\ \hline
IEEE Xplore        & 153                   \\ \hline
ScienceDirect      & 677                   \\ \hline
\end{tabular}
\caption{Hakutulosten määrä hakukohteista}
\label{table: hakutulokset}
\end{table}

\section{Aineiston valintakriteerit} \label{valintakriteerit}
Tutkimuksen aineistoon pyritään keräämään tutkmuksia, joissa jonain tutkimuksen metodeista, joko pää- tai sivumetodina, on käytetty laadullista simulaatiota.

Kriteerit tutkimuksen aineistolle sopivalle materiaalille ovat:
\begin{enumerate}
    \item Artikkeli on akateeminen julkaisu tai konferenssipaperi
    \item Artikkeli on saatavilla englanniksi tai suomeksi.
    \item Artikkeli on saatavilla digitaalisessa muodossa.
    \item Artikkeli on saatavilla kokonaisuudessaan.
    \item Artikkelin menetelmissä on käytetty laadullista simulaatioita.
\end{enumerate}

Artikkelin materiaalista poissulkevat kriteerit;
\begin{enumerate}
    \item Artikkelissa vain viitataan toisen tutkimuksen toteuttamaan simulaatioon, eikä simulaatio ole ollut osana tutkimuksen menetelmiä.
    \item Artikkelin toteuttama simulaatio ei ole tietokonesimulaatio.
    \item Artikkeli on laadullinen tutkimus simulaatioiden käytöstä tai kokemuksista.
\end{enumerate}

\section{Aineiston valintaprosessi} \label{valintaprosessi}
Aineisto kerättiin Mendeley-viittaushallintaohjelmistolla\footnote{\url{https://www.mendeley.com/}}. Mendeley-ohjelmalla pystyttiin keräämään artikkelit suoraan verkkosivun kautta käyttäen ohjelman webselaimen laajennusta. Ohjelmalla pystyttiin tallentamaan automaattisesti viittaus valituista artikkeleista, sisältäen artikkelin tekijät, julkaisuvuoden, otsikon, julkaisun lähteen ja milloin on artikkeli on lisätty viittauskantaan. Artikkelien sopivuus katsottiin luvun \ref{valintakriteerit} kriteerien perusteella. Artikkelien valinta seurasi usein seuraavia askeleita:

\begin{enumerate}
    \item Viittaako artikkelin otsikko laadullisen simulaation käyttöstä?
    \item Mainitseeko tiivistelmä toteutetusta laadullisesta simulaatiosta?
    \item Onko artikkelin metodeissa mainittu laadullisen simulaation käyttö?
\end{enumerate}

Ensimmäisestä 677 artikkelista sopiviksi valittiin 244. Näistä tarkemmin katsottuna karsittiin pois 53 artikkelia, eli aineistoon jäi lopuksi 191 artikkelia.

\section{Tiedon kerääminen aineistosta} \label{aineiston analysointi}
Ensimmäistä tutkimuskysymystä varten jokaiseen artikkeliin merkittiin tagilla tieteenala, mitä artikkelin tutkimus koski. Mendeley-ohjelmalla pystyttiin tagien kautta katsomaan kuinka monta artikkelia oli per ala, jotka merkittiin taulukkoon. Toista tutkimuskysymystä varten artikkeleista haettiin viittauksia ohjeisiin ja työkaluihin. Löydetyt ohjeet ja työkalut merkittiin artikkeleiden tageihin, jotka koostettiin taulukkoon.

Kolmasta tutkimuskysymystä varten artikkeleista merkittiin lyhyt kooste mitä tietoa tai lopputulosta oltiin käytetyllä laadullisella simulaatiolla etsitty. Koosteista etsittiin yhteensopivuuksia kategorisointia varten. Sopivat kategoriat kerättiin taulokkoon, johon merkittiin artikkelien lukumäärät. Tämän metodin testaamista varten tutkimusaineistosta otettiin ensimmäiseksi satunnaisesti 20 artikkelia, joista kerättiin tieto, mitä tuloksia laadullisilla simulaatioilla haluttiin. Tämän jälkeen etsittiin yhtäläisyykset ja näistä saatiin alustavat kategoriat. Luvun \ref{simulaatiotulokset} taulukosta \ref{table: tuloksien kategoriat} on esitetty tutkimusten tuloksista löydetyt kategoriat.

\chapter{Tulokset}
Tässä kappaleessa esitellään kerätyn aineiston lopputulokset. Luvussa \ref{tieteenalat} esitellään tutkimusartikkeleiden päätieteenalat ja niiden lukumäärät. Luku \ref{tyokalut} käy läpi artikkeleista löydetyt ja tunnistetut algoritmit, ohjemistot ja ohjelmistokehykset. Luku \ref{simulaatiotulokset} kuvailee tutkimusartikkeleista löydetyt kategoriat ja niiden lukumäärät.

\section{Tieteenalat} \label{tieteenalat}
Löydetyt tieteen alat ja artikkelien lukumäärät aloittan on esitetty taulukoissa insinööri tieteet \ref{table:engineering} (eng. engineering sciences), luonnontieteet \ref{table:naturalscience} (eng. natural sciences) ja sosiaali- sekä humanistitieteet\ref{table:socialscience} (eng. social and humanistic sciences).

\begin{table}[htbp]
\centering
\begin{tabular}{ll}
\toprule
\textbf{Ala} & \textbf{lkm} \\
\midrule
Automotive Engineering & 1 \\
Building Science & 3 \\
Chemical Engineering & 24 \\
Computer Science & 12 \\
Electronics & 7 \\
Materials Science & 17 \\
Mechanical Systems & 3 \\
Nanotechnology & 1 \\
Nuclear Science & 3 \\
Robotics & 2 \\
\bottomrule
\end{tabular}
\caption{engineering sciences}
\label{table:engineering}
\end{table}


\begin{table}[htbp]
\centering
\begin{tabular}{ll}
\toprule
\textbf{Ala} & \textbf{lkm} \\
\midrule
Agricultural Science & 4 \\
Anatomy & 5 \\
Astronomy & 1 \\
Bacteríology & 5 \\
Biology & 17 \\
Chemistry & 21 \\
Ecology & 14 \\
Enviromental Science & 10 \\
Geology & 3 \\
Pathology & 2 \\
Physics & 10 \\
\bottomrule
\end{tabular}
\caption{natural sciences}
\label{table:naturalscience}
\end{table}


\begin{table}[htbp]
\centering
\begin{tabular}{ll}
\toprule
\textbf{Ala} & \textbf{lkm} \\
\midrule
Behavioral Science & 4 \\
Control Theory & 1\\
Economy & 4 \\
Epidemiology & 1 \\
Psychology & 3 \\
Systems Engineering & 2 \\
Transport Science & 2 \\
Urban Studies & 4 \\
\bottomrule
\end{tabular}
\caption{social and humanistic sciences}
\label{table:socialscience}
\end{table}


\clearpage
\section{Laadullisten simulaatioiden ohjeet ja työkalut} \label{tyokalut}
Seuraavissa listoissa on esitetty tutkimuksista tunnistetut työkalut ja ohjeistukset laadullisten simulaatioiden luomiseen. Nämä ovat jaettu kahteen kategoriaan, ohjelmistot ja mallintamisen työkalut. Kategoriaan "ohjelmistot" kuuluvat tietokoneohjelmat ja ohjelmistot joiden avulla voidaan joko luoda tai ajaa laadullisia simulaatioita.
Kategoriaan "mallintaminen" kuuluvat metodit ja algoritmit joiden avulla voidaan mallintaa laadullisia malleja.

\subsection{Mallintaminen (TODO: jokin tarkempi jäsentely)}
\begin{itemize}
    \item hazard identification method
    \item HAZOP
    \item fuzzy systems 
    \item finite element method 
    \item QSIM 
    \item fuzzy qualitative simulation algorithm 
    \item grey qualitative constraint filtering method 
    \item GA-BP (Genetic Algorithm – Back Propagation Neural Network Algorithm) 
    \item Morven framework 
    \item Biomolecular Network Ontology 
    \item agent-based approach 
    \item Qualitative Archi Bond Graphs 
    \item finite element method 
    \item fuzzy qualitative simulation algorithm 
    \item signed directed graph 
    \item extended signed directed graph 
    \item fault-tree structures 
    \item qualitative trend analysis algorithm 
    \item optimum discretization method 
    \item form particle difference method 
    \item computational fluid dynamics 
    \item discrete element method 
    \item Dynamic trend analysis 
    \item SQMA (Situation-based Qualitative Modeling and Analysis) 
    \item qualitative fault isolation (QFI) framework 
    \item Catastrophe Theory 
    \item stochastic social choice rules 
    \item QUAL2Kw 
    \item multi-objective modified seagull optimization 
    \item lattice Boltzmann method 
    \item DEVS formalism 
    \item Semantic Micro Object Language (SMOL) based modeling 
    \item frozen phonon approach 
    \item Nose–Hoover dynamics 
    \item System Thermal-Hydraulic (STH) analysis 
    \item fuzzy set theory 
    \item fuzzy qualitative reasoning 
    \item agent-based architecture 
    \item multi-objective particle swarm optimization (MOPSO) algorithm 
    \item Causal Loop Diagrams 
    \item slime mould algorithm 
    \item Euler-Euler approach
\end{itemize}


\subsection{Ohjelmistot}
Ohjelmistot on luokiteltu toteutettuihin simulaatioiden lähestymistapoihin. Mallintamisessa auttavat ohjelmat on myös merkitty omaksi luokaksi.

Agent-based simulations:
\begin{itemize}
    \item Repast
    \item SWARM
    \item Presage2
\end{itemize}

Computational fluid dynamics:
\begin{itemize}
    \item OpenFOAM
    \item FLUENT
    \item Flow3D
\end{itemize}

Continuous simulations:
\begin{itemize}
    \item AnyLogic
    \item Matlab
    \item Maplesoft
    \item WINEPR-Simfonia
    \item SCAPS 1D
    \item Pace3D
    \item DIOS
\end{itemize}

Discrete Element Simulations:
\begin{itemize}
    \item LIGGGHTS 
    \item OVITO
    \item ns-2 simulator
    \item GARP
    \item Garp3
    \item Idrisi GIS
    \item SABER
    \item Matlab
    \item V-REP
\end{itemize}

Finite element simulations:
\begin{itemize}
    \item ANSYS
    \item Matlab
    \item COMSOL
    \item Poynting
    \item DEAL.II
\end{itemize}

Gene network simulations:
\begin{itemize}
    \item GNA (Genetic Network Analyzer) 
    \item GRENS (Gene REgulatory Network Simulator)
    \item GINsim (Gene Interaction Network simulation)
    \item Matlab
    \item CellNetAnalyzer
\end{itemize}


Molecyclar dynamics simulations:
\begin{itemize}
    \item LAMMPS
    \item Materials Studio
    \item GROMACS 4.5
\end{itemize}

Others:
\begin{itemize}
    \item Matlab (Signed diagraph simulations)
    \item ReaDDy 2 simulator (Interacting particle reaction dynamic simulations)
    \item Chemkin-Pro (Simulating decomposition pathways in the bimolecular reactions)
    \item HyperChem 8.0 (Molecular mechanics simulator)
    \item GMS-7 software (Simulating effects of constructing a hydraulic removal system that prevents entry and penetration of water into an aquifer)
    \item LUX (Ray-tracing Monte-Carlo program)
\end{itemize}

Mallintamisessa auttavat ohjelmat:
\begin{itemize}
    \item TAM-B (biological reaction systems)
    \item UPPAAL (gene regulatory networks)
    \item TAM-C (structures and parameters of formal kinetics)
    \item SIMULINK (chemical engineering fault propagation scenarios)
    \item ThermRot (calculating bimolecular thermodynamic and kinetic data)
    \item EcoMata (ecosystems and their behaviours)
    \item MAGMAS3D (Model for the Analysis of General Multilayered plAnar and 3D Structures)
    \item LASS (modelling environment for generating landscapes and analysing landscape patterns and simulation results)
    \item EPANET (tool for water distribution system analysis)
    \item Visual-FIR (modelling and simulation of complex systems based on Fuzzy Inductive Reasoning)
    \item SQSIM (simulator that predicts semi-quantitative behavior descriptions from the  semi-quantitative differential equations)
\end{itemize}


\section{Etsityt tulokset simulaatioista} \label{simulaatiotulokset}
Seuraava lista sisältää selitykset taulukon \ref{table: tuloksien kategoriat} kategoriolle.
Tutkimuksen etsityt tulokset voivat sopia myös useampaan eri kategoriaan samaan aikaan. 

\begin{enumerate}
    \item Artikkelit viittaavat tutkimuksiin missä pyritään selvittämään mitä tapahtuu jos järjestelmässä tapahtuu jotain epätoivottua. Esimerkiksi tutkimuksessa \cite{hu2015cusp} tutkittiin mitä organisaatiossa voi tapahtua jos työntekijä poistuu organisaatiosta äkillisesti.
    \item Tutkimuksissa etsitään mahdollisia vikojen syitä tai niiden mahdollisuuksia. Usein tästä puhutaan termillä \textit{ vian diagnoosi (eng. fault diagnosis)}. Vian diagnoosia on mahdollista toteuttaa laadullisen simulaation avulla.
    \item Tutkimukset, joissa haluttiin vähentää tutkimuskohteen tiedon aukkoja simulaation avulla, esim. miten kohteen eri ominaisuudet vaikuttavat toisiinsa tai ylipäätänsi miksi saavuttiin lopputulokseen.
    \item Tutkimukset joissa käytettiin 2 tai useampaa menetelmää, joista yksi oli laadullinen simulaatio. Tutkimuksissa verrattiin muista menetelmistä saatuja tuloksia laadullisen simulaation saamaan ja pyrittiin varmentamaan joko simulaation tarkkuutta tai tulosten oikeellisuutta.
    \item Tutkimukset missä haluttiin katsoa miten olosuhteiden muuttuminen vaikuttaa tutkimuskohteeseen, esimerkiksi miten veden saasteisuus vaikuttaa vesistön kalakantaan.
    \item Tutkimukset, joissa haluttiin selvittää miten parantaa jotain tiettyä tapaa tai metodia, esimerkiksi tutkimuksessa \textcite{zhang2015automatic} luotiin työkalu HAZOP, jolla voidaan automatisoida kemiallisen tuotannon vaarakohteiden löytämistä. Yksi HAZOP:in käyttämistä menetelmistä on laadullinen simulaatio.
    \item Tutkimukset kohdistuivat tutkimuksiin, joissa pyritiin luomaan työkaluja tai metodeja juuri laadullisten simulaatioiden luontiin.
    \item Tutkimukset, joissa käytiin läpi useampaa eri laadullista simulaatioita tai malleja ja näiden toimintaa verrattiin keskenään.
    \item Tutkimukset, joissa laadullisella simulaatiolla haluttiin ennustaa järjestelmän toimintaa.
\end{enumerate}


\begin{table}[]
\resizebox{\textwidth}{!}{%
\begin{tabular}{|l|l|}
\hline
Kategoriat                                                                                                       & Artikkelien lukumäärä \\ \hline
1.Pyritään selvittämään miten toteutunut ongelma/haitta vaikuttaa kohteeseen tai tilanteeseen.                   &         5              \\ \hline
2.Pyritään selvittämään vian syitä/mahdollisuuksia.                                                              &           27            \\ \hline
3.Pyritään vähentämään tutkimuskohteen tiedon aukkoja.                                                           &           52            \\ \hline
4.Simulaatiolla pyritään vertaamaan tuloksia toiseen tutkimukseen tai samassa tutkimuksessa tehtyihin kokeisiin. &           42            \\ \hline
5.Halutaan selvittää miten olosuhteiden vaihtuminen vaikuttaa simulaation kohteeseen.                            &          21             \\ \hline
6.Halutaan optimoida jotain metodeja/tapoja liittyen tutkittavaan kohteeseen/työkaluun.                          &          36             \\ \hline
7.Tutkimuksessa luotiin kokonaan uusi työkalu/metodi laadullisten simulaatioiden luontiin/käyttämiseen.          &           36            \\ \hline
8.Arvioidaan erilaisten laadullisten simulaatioiden/algoritmien käyttöä.                                         &           5            \\ \hline
9. Halutaan ennustaa kohteen käyttäytymistä/toimintaa.                                                           &           7            \\ \hline
\end{tabular}%
}
\label{table: tuloksien kategoriat}
\caption{Tuloksien kategoriat.}
\end{table}

\chapter{Tulosten tulkitseminen}


\chapter{Yhteenveto}


\printbibliography

\appendix

\section{Aineistossa käytetyt tutkimukset}
\textbf{TODO: miten saadaan viittaukset pysymään nätisti sivun rajoissa?}

Abdul-Wahab, S A, A Elkamel, M A Al-Weshahi, and A S al Yahmadi. 2007. “Troubleshooting the Brine Heater of the MSF Plant Fuzzy Logic-Based Expert System.” Desalination 217 (1): 100–117. https://doi.org/https://doi.org/10.1016/j.desal.2007.01.014.

Ahmad, Jamil, Jérémie Bourdon, Damien Eveillard, Jonathan Fromentin, Olivier Roux, and Christine Sinoquet. 2009. “Temporal Constraints of a Gene Regulatory Network: 
Refining a Qualitative Simulation.” Biosystems 98 (3): 149–59.
https://doi.org/https://doi.org/10.1016/j.biosystems.2009.05.002.

Aichernig, Bernhard K, Harald Brandl, and Franz Wotawa. 2009. “Conformance Testing of Hybrid Systems with Qualitative Reasoning Models.” Electronic Notes in Theoretical Computer Science 253 (2): 53–69. https://doi.org/https://doi.org/10.1016/j.entcs.2009.09.051.

Argani, L P, D Misseroni, A Piccolroaz, Z Vinco, D Capuani, and D Bigoni. 2016. “Plastically-Driven Variation of Elastic Stiffness in Green Bodies during Powder Compaction: Part I. Experiments and Elastoplastic Coupling.” Journal of the European Ceramic Society 36 (9): 2159–67. https://doi.org/https://doi.org/10.1016/j.jeurceramsoc.2016.02.012.

Aristov, V v, A A Frolova, S A Zabelok, R R Arslanbekov, and V I Kolobov. 2012. “Simulations of Pressure-Driven Flows through Channels and Pipes with Unified Flow Solver.” Vacuum 86 (11): 1717–24. https://doi.org/https://doi.org/10.1016/j.vacuum.2012.02.043.

Arroyo, Esteban, Mario Hoernicke, Pablo Rodríguez, and Alexander Fay. 2016. “Automatic Derivation of Qualitative Plant Simulation Models from Legacy Piping and Instrumentation Diagrams.” Computers \& Chemical Engineering 92: 112–32. https://doi.org/https://doi.org/10.1016/j.compchemeng.2016.04.040.

Ayadi, Ali, Cecilia Zanni-Merk, François de Bertrand de Beuvron, Julie Thompson, and Saoussen Krichen. 2019. “BNO—An Ontology for Understanding the Transittability of Complex Biomolecular Networks.” Journal of Web Semantics 57: 100495. https://doi.org/https://doi.org/10.1016/j.websem.2019.01.002.

Baldazzi, Valentina, Delphine Ropers, Johannes Geiselmann, Daniel Kahn, and Hidde de Jong. 2012. “Importance of Metabolic Coupling for the Dynamics of Gene Expression Following a Diauxic Shift in Escherichia Coli.” Journal of Theoretical Biology 295: 100–115. https://doi.org/https://doi.org/10.1016/j.jtbi.2011.11.010.

Barbari, Matteo, Alberto Cavalli, Lorenzo Fiorineschi, Massimo Monti, and Marco Togni. 2014. “Innovative Connection in Wooden Trusses.” Construction and Building Materials 66: 654–63. https://doi.org/https://doi.org/10.1016/j.conbuildmat.2014.06.022.

Barnat, J, L Brim, I Černá, S Dražan, J Fabriková, and D Šafránek. 2009. “On Algorithmic Analysis of Transcriptional Regulation by LTL Model Checking.” Theoretical Computer Science 410 (33): 3128–48. https://doi.org/https://doi.org/10.1016/j.tcs.2009.02.017.

Barnat, J, L Brim, I Černá, S Dražan, and D Šafránek. 2008. “Parallel Model Checking Large-Scale Genetic Regulatory Networks with DiVinE.” Electronic Notes in Theoretical Computer Science 194 (3): 35–50. https://doi.org/https://doi.org/10.1016/j.entcs.2007.12.001.

Bellazzi, R, R Guglielmann, and L Ironi. 2001. “Learning from Biomedical Time Series through the Integration of Qualitative Models and Fuzzy Systems.” Artificial Intelligence in Medicine 21 (1): 215–20. https://doi.org/https://doi.org/10.1016/S0933-3657(00)00088-9.

Bellazzi, Riccardo, Raffaella Guglielmann, Liliana Ironi, and Cesare Patrini. 2001. “A Hybrid Input-Output Approach to Model Metabolic Systems: An Application to Intracellular Thiamine Kinetics.” Journal of Biomedical Informatics 34 (4): 221–48. https://doi.org/https://doi.org/10.1006/jbin.2001.1022.

Bernard, Olivier, and Jean-Luc Gouzé. 2002. “Global Qualitative Description of a Class of Nonlinear Dynamical Systems.” Artificial Intelligence 136 (1): 29–59. https://doi.org/https://doi.org/10.1016/S0004-3702(01)00169-2.

Bourgeat, Johan, and Philippe Galy. 2013. “Single and Compact ESD Device Beta-Matrix Solution Based on Bidirectional SCR Network in Advanced 28/32 Nm Technology Node.” Solid-State Electronics 87: 34–42. https://doi.org/https://doi.org/10.1016/j.sse.2013.04.033.

Brown, Alexander L, and Walter F Dabberdt. 2003. “Modeling Ventilation and Dispersion for Covered Roadways.” Journal of Wind Engineering and Industrial Aerodynamics 91 (5): 593–608. https://doi.org/https://doi.org/10.1016/S0167-6105(02)00472-5.

Bulitko, Vadim, and David C Wilkins. 2003. “Qualitative Simulation of Temporal Concurrent Processes Using Time Interval Petri Nets.” Artificial Intelligence 144 (1): 95–124. https://doi.org/https://doi.org/10.1016/S0004-3702(02)00369-7.

Calado, J M F, F.P.N.F. Carreira, M.J.G.C. Mendes, J M G Sá da Costa, and M Bartys. 2003. “Fault Detection Approach Based on Fuzzy Qualitative Reasoning Applied to the DAMADICS Benchmark Problem.” IFAC Proceedings Volumes 36 (5): 1077–82. https://doi.org/https://doi.org/10.1016/S1474-6670(17)36636-3.

Calado, J M F, J M G Sá da Costa, M Bartys, and J Korbicz. 2006. “FDI Approach to the DAMADICS Benchmark Problem Based on Qualitative Reasoning Coupled with Fuzzy Neural Networks.” Control Engineering Practice 14 (6): 685–98. https://doi.org/https://doi.org/10.1016/j.conengprac.2005.03.025.

Canale, Giacomo, Felice Rubino, and Roberto Citarella. 2024. “Design Aspects of a CMC Coating-like System for Hot Surfaces of Aero Engine Components.” Forces in Mechanics 14: 100251. https://doi.org/https://doi.org/10.1016/j.finmec.2023.100251.

Cao, Qinggui, Kai Yu, Lujie Zhou, Linlin Wang, and Chunai Li. 2019. “In-Depth Research on Qualitative Simulation of Coal Miners’ Group Safety Behaviors.” Safety Science 113: 210–32. https://doi.org/https://doi.org/10.1016/j.ssci.2018.11.012.

Chan, H Y, M P Srinivasan, F Benistant, K R Mok, Lap Chan, and H M Jin. 2006. “Continuum Modeling of Post-Implantation Damage and the Effective plus Factor in Crystalline Silicon at Room Temperature.” Thin Solid Films 504 (1): 269–73. https://doi.org/https://doi.org/10.1016/j.tsf.2005.09.167.

Chang, Chuei-Tin, and Jung-Yang Chen. 2009. “Systematic Development of Automata Generated Languages for Fault Diagnosis in Continuous Chemical Processes.” IFAC Proceedings Volumes 42 (11): 303–8. https://doi.org/https://doi.org/10.3182/20090712-4-TR-2008.00047.

Chang, Sheng-Yung, Cheng-Ren Lin, and Chuei-Tin Chang. 2002. “A Fuzzy Diagnosis Approach Using Dynamic Fault Trees.” Chemical Engineering Science 57 (15): 2971–85. https://doi.org/https://doi.org/10.1016/S0009-2509(02)00178-1.

Chen, Jung Yang, and Chuei-Tin Chang. 2009. “Development of Fault Diagnosis Strategies Based on Qualitative Predictions of Symptom Evolution Behaviors.” Journal of Process Control 19 (5): 842–58. https://doi.org/https://doi.org/10.1016/j.jprocont.2008.11.006.

Chen, Tao, Xiaoke Ku, Jianzhong Lin, and Hanhui Jin. 2019. “Modelling the Combustion of Thermally Thick Biomass Particles.” Powder Technology 353: 110–24. https://doi.org/https://doi.org/10.1016/j.powtec.2019.05.011.

Chen, Wei, Jianyong Zhang, Timothy Donohue, Kenneth Williams, Ruixue Cheng, Mark Jones, and Bin Zhou. 2017. “Effect of Particle Degradation on Electrostatic Sensor Measurements and Flow Characteristics in Dilute Pneumatic Conveying.” Particuology 33: 73–79. https://doi.org/https://doi.org/10.1016/j.partic.2016.10.004.

Chen, Zhiyuan, and Tiemin Li. 2023. “Precise Geometry and Pose Measurement of In-Hand Objects with Simple Features Using a Multi-Camera System.” Manufacturing Letters 35: 40–48. https://doi.org/https://doi.org/10.1016/j.mfglet.2023.07.020.

Chipaux, R, and M Géléoc. 2000. “Modelisation and Simulation of the Light Collection in the CMS Lead Tungstate Crystals.” Nuclear Instruments and Methods in Physics Research Section A: Accelerators, Spectrometers, Detectors and Associated Equipment 451 (3): 610–22. https://doi.org/https://doi.org/10.1016/S0168-9002(00)00334-X.

Cui, Yunqiu, Chunjie Niu, Jianhua Lv, Hongyu Fan, Chao Chen, Dongping Liu, Na Lu, Guangjiu Lei, and Weiyuan Ni. 2023. “Temperature Dependence of Surface Microstructure of 316 L Stainless Steel Exposed to Low-Energy Hydrogen Ion Irradiation.” Fusion Engineering and Design 196: 113976. https://doi.org/https://doi.org/10.1016/j.fusengdes.2023.113976.

Dafalias, Yannis F, and Majid T Manzari. 2024. “Rate Dependent but Time Independent Plasticity Formulation with Application to Sand.” Mechanics Research Communications 135: 104242. https://doi.org/https://doi.org/10.1016/j.mechrescom.2023.104242.

Daigle, Matthew J, Anibal Bregon, Xenofon Koutsoukos, Gautam Biswas, and Belarmino Pulido. 2016. “A Qualitative Event-Based Approach to Multiple Fault Diagnosis in Continuous Systems Using Structural Model Decomposition.” Engineering Applications of Artificial Intelligence 53: 190–206. https://doi.org/https://doi.org/10.1016/j.engappai.2016.04.002.

Davidsson, Paul, and Magnus Boman. 2005. “Distributed Monitoring and Control of Office Buildings by Embedded Agents.” Information Sciences 171 (4): 293–307. https://doi.org/https://doi.org/10.1016/j.ins.2004.09.007.

Dias, R I, P Salles, and R H Macedo. 2009. “Mate Guarding and Searching for Extra-Pair Copulations: Decision-Making When Interests Diverge.” Ecological Informatics 4 (5): 405–12. https://doi.org/https://doi.org/10.1016/j.ecoinf.2009.09.008.

Dou, Liangtan, Ying Sun, and Lian She. 2012. “Research on Efficiency of Collaborative Allocation System of Emergency Material Based on Synergetic Theory.” Systems Engineering Procedia 5: 240–47. https://doi.org/https://doi.org/10.1016/j.sepro.2012.04.038.

Drăgoicea, Monica, João Falcão e Cunha, and Monica Pătraşcu. 2015. “Self-Organising Socio-Technical Description in Service Systems for Supporting Smart User Decisions in Public Transport.” Expert Systems with Applications 42 (17): 6329–41. https://doi.org/https://doi.org/10.1016/j.eswa.2015.04.029.

Dubi, Shay, Chen Dubi, and Yonatan Dubi. 2007. “A Two Phase Harmonic Model for Left Ventricular Function.” Medical Engineering \& Physics 29 (9): 984–88. https://doi.org/https://doi.org/10.1016/j.medengphy.2006.10.015.

Edwards, Roderick, and Liliana Ironi. 2014. “Periodic Solutions of Gene Networks with Steep Sigmoidal Regulatory Functions.” Physica D: Nonlinear Phenomena 282 (July): 1–15. https://doi.org/10.1016/J.PHYSD.2014.04.013.

Eggeman, A S, A London, and P A Midgley. 2013. “Ultrafast Electron Diffraction Pattern Simulations Using GPU Technology. Applications to Lattice Vibrations.” Ultramicroscopy 134: 44–47. https://doi.org/https://doi.org/10.1016/j.ultramic.2013.05.013.

Eisenack, K, and J Kropp. 2001. “Assessment of Management Options in Marine Fisheries by Qualitative Modelling Techniques.” Marine Pollution Bulletin 43 (7): 215–24. https://doi.org/https://doi.org/10.1016/S0025-326X(01)00076-5.

Eisenack, K, H Welsch, and J P Kropp. 2006. “A Qualitative Dynamical Modelling Approach to Capital Accumulation in Unregulated Fisheries.” Journal of Economic Dynamics and Control 30 (12): 2613–36. https://doi.org/https://doi.org/10.1016/j.jedc.2005.08.004.

Escobet, Antoni, Àngela Nebot, and Francisco Mugica. 2014. “PEM Fuel Cell Fault Diagnosis via a Hybrid Methodology Based on Fuzzy and Pattern Recognition Techniques.” Engineering Applications of Artificial Intelligence 36: 40–53. https://doi.org/https://doi.org/10.1016/j.engappai.2014.07.008.

Eskandaripour, Mehrtash, Mohammad H Golmohammadi, and Shahrokh Soltaninia. 2023. “Optimization of Low-Impact Development Facilities in Urban Areas Using Slime Mould Algorithm.” Sustainable Cities and Society 93: 104508. https://doi.org/https://doi.org/10.1016/j.scs.2023.104508.

Eskandaripour, Mehrtash, and Shahrokh Soltaninia. 2024. “Optimal Low-Impact Development Facility Design in Urban Environments: A Multidimensional Optimization Approach Employing Slime Mould and Nondominated Sorting Genetic Algorithms.” Urban Climate 55: 101963. https://doi.org/https://doi.org/10.1016/j.uclim.2024.101963.

Evdochenko, E, A Kalde, J di Ronco, K Albert, J Kamp, and Matthias Wessling. 2024. “Unraveling the Ion Transport through Top and Wall Coated Polyelectrolyte Membrane Pores.” Desalination, 118170. https://doi.org/https://doi.org/10.1016/j.desal.2024.118170.

Fan, Guo-Wei, and Xue-Feng Li. 2024. “Hypoplastic Constitutive Model of Inherently Anisotropic Sand.” Computers and Geotechnics 175: 106645. https://doi.org/https://doi.org/10.1016/j.compgeo.2024.106645.

Feng, Yahui, Hongxia Guo, Wuying Ma, Jiawen Hu, Xiaoping Ouyang, Jinxin Zhang, Fengqi Zhang, et al. 2024. “Effect of 60Coγ Ray Radiation on Electrical Properties of SiGe HBTs at Low Temperatures.” Microelectronics Journal 144: 106058. https://doi.org/https://doi.org/10.1016/j.mejo.2023.106058.

Fernandes, Eduarda, Irene López-Sicilia, Maria Teresa Martín-Romero, Juan Giner-Casares, and Marlene Lúcio. 2024. “A Stratum Corneum Lipid Model as a Platform for Biophysical Profiling of Bioactive Chemical Interactions at the Skin Level.” Journal of Molecular Liquids 400: 124513. https://doi.org/https://doi.org/10.1016/j.molliq.2024.124513.

Galuza, A, I Kolenov, D Vinnikov, S Mizrakhy, and A Savchenko. 2022. “Investigation of Micro-Arc Oxidation Coatings Using Sub-THz Ellipsometry.” Materials Characterization 189: 111930. https://doi.org/https://doi.org/10.1016/j.matchar.2022.111930.

GAO, Dong, Chongguang WU, Beike ZHANG, and Xin MA. 2010. “Signed Directed Graph and Qualitative Trend Analysis Based Fault Diagnosis in Chemical Industry.” Chinese Journal of Chemical Engineering 18 (2): 265–76. https://doi.org/https://doi.org/10.1016/S1004-9541(08)60352-3.

Gelman, Irit Askira. 2005. “Addressing Time-Scale Differences among Decision-Makers through Model Abstractions.” European Journal of Operational Research 160 (2): 325–35. https://doi.org/https://doi.org/10.1016/j.ejor.2003.09.004.

Ghafari, Sara, Mohammad Ebrahim Banihabib, and Saman Javadi. 2020. “A Framework to Assess the Impact of a Hydraulic Removing System of Contaminate Infiltration from a River into an Aquifer (Case Study: Semnan Aquifer).” Groundwater for Sustainable Development 10: 100301. https://doi.org/https://doi.org/10.1016/j.gsd.2019.100301.

Ghasemi, M A, and S R Falahatgar. 2020. “Qualitative Description on Channel Cracking in Ceramic Type Coatings Due to Substrate Tension Using 3D Discrete Element Method.” Ceramics International 46 (5): 5548–65. https://doi.org/https://doi.org/10.1016/j.ceramint.2019.08.198.

Goethem, S van, J.-M. Jacquet, L Brim, and D Šafránek. 2013. “Timed Modelling of Gene Networks with Arbitrarily Precise Expression Discretization.” Electronic Notes in Theoretical Computer Science 293: 67–81. https://doi.org/https://doi.org/10.1016/j.entcs.2013.02.019.

Goulart, Fernando F, Paulo Salles, Carlos H Saito, and Ricardo B Machado. 2013. “How Do Different Agricultural Management Strategies Affect Bird Communities Inhabiting a Savanna-Forest Mosaic? A Qualitative Reasoning Approach.” Agriculture, Ecosystems \& Environment 164: 114–30. https://doi.org/https://doi.org/10.1016/j.agee.2012.09.011.

Guerrin, F, and J Dumas. 2001. “Knowledge Representation and Qualitative Simulation of Salmon Redd Functioning. Part II: Qualitative Model of Redds.” Biosystems 59 (2): 85–108. https://doi.org/https://doi.org/10.1016/S0303-2647(01)00101-0.

Guerrin, François, and Jacques Dumas. 2001. “Knowledge Representation and Qualitative Simulation of Salmon Redd Functioning. Part I: Qualitative Modeling and Simulation.” Biosystems 59 (2): 75–84. https://doi.org/https://doi.org/10.1016/S0303-2647(01)00100-9.

Gueven, A, and Z Hicsasmaz. 2011. “Geometric Network Simulation of High Porosity Foods.” Applied Mathematical Modelling 35 (10): 4824–40. https://doi.org/https://doi.org/10.1016/j.apm.2011.03.047.

Gutiérrez, M A, H Khanbareh, and S van der Zwaag. 2016. “Computational Modeling of Structure Formation during Dielectrophoresis in Particulate Composites.” Computational Materials Science 112: 139–46. https://doi.org/https://doi.org/10.1016/j.commatsci.2015.10.011.

Hamagchi, Takashi, Yoshihiro Hashimoto, Toshiaki Itoh, Akihiko Yoneya, and Yoshitaka Togari. 2001. “Abnormal Situation Correction Based on Cascade/Ratio Control.” IFAC Proceedings Volumes 34 (27): 209–14. https://doi.org/https://doi.org/10.1016/S1474-6670(17)33593-0.

Hasan, Salim Ibrahim, Serhan Küçüka, and Mehmet Akif Ezan. 2023. “Cooling Performance of a Piezo-Fan Oscillating in a Vertical Channel with Natural Convection.” International Communications in Heat and Mass Transfer 141: 106602. https://doi.org/https://doi.org/10.1016/j.icheatmasstransfer.2022.106602.

Honorien, J, R Fournet, P.-A. Glaude, and B Sirjean. 2021. “Theoretical Study of the Gas-Phase Thermal Decomposition of Urea.” Proceedings of the Combustion Institute 38 (1): 355–64. https://doi.org/https://doi.org/10.1016/j.proci.2020.06.012.

Hu, Bin, and Ni Xia. 2015a. “Cusp Catastrophe Model for Sudden Changes in a Person’s Behavior.” Information Sciences 294: 489–512. https://doi.org/https://doi.org/10.1016/j.ins.2014.09.055.

———. 2015b. “Cusp Catastrophe Model for Sudden Changes in a Person’s Behavior.” Information Sciences 294 (February): 489–512. https://doi.org/10.1016/J.INS.2014.09.055.

Hussain, Raja Rizwan, and Tetsuya Ishida. 2009. “Critical Carbonation Depth for Initiation of Steel Corrosion in Fully Carbonated Concrete and Development of Electrochemical Carbonation Induced Corrosion Model.” International Journal of Electrochemical Science 4 (8): 1178–95. https://doi.org/https://doi.org/10.1016/S1452-3981(23)15216-3.

Ironi, Liliana, and Diana X Tran. 2016. “Nonlinear and Temporal Multiscale Dynamics of Gene Regulatory Networks: A Qualitative Simulator.” Mathematics and Computers in Simulation 125: 15–37. https://doi.org/https://doi.org/10.1016/j.matcom.2015.11.007.

Jagannathan, N Suhas, and Lisa Tucker-Kellogg. 2016. “Membrane Permeability during Pressure Ulcer Formation: A Computational Model of Dynamic Competition between Cytoskeletal Damage and Repair.” Journal of Biomechanics 49 (8): 1311–20. https://doi.org/https://doi.org/10.1016/j.jbiomech.2015.12.022.

Jong, Hidde de, Johannes Geiselmann, Grégory Batt, Céline Hernandez, and Michel Page. 2004. “Qualitative Simulation of the Initiation of Sporulation in Bacillus Subtilis.” Bulletin of Mathematical Biology 66 (2): 261–99. https://doi.org/https://doi.org/10.1016/j.bulm.2003.08.009.

Jong, Hidde de, Jean Luc Gouzé, Céline Hernandez, Michel Page, Tewfik Sari, and Johannes Geiselmann. 2004. “Qualitative Simulation of Genetic Regulatory Networks Using Piecewise-Linear Models.” Bulletin of Mathematical Biology 66 (2): 301–40. https://doi.org/10.1016/J.BULM.2003.08.010.

Ju, Shi-Ning, Cheng-Liang Chen, and Chuei-Tin Chang. 2003. “Fault-Tree Structures of Override Control Systems.” Reliability Engineering \& System Safety 81 (2): 163–81. https://doi.org/https://doi.org/10.1016/S0951-8320(03)00086-3.

Junior, Flávio Neves, and Joseph Aguilar Martin. 2000. “Heterogeneous Control and Qualitative Supervision, Application to a Distillation Column.” Engineering Applications of Artificial Intelligence 13 (2): 179–97. https://doi.org/https://doi.org/10.1016/S0952-1976(99)00051-2.

Jurado, Sergio, Àngela Nebot, Fransisco Mugica, and Narcís Avellana. 2015. “Hybrid Methodologies for Electricity Load Forecasting: Entropy-Based Feature Selection with Machine Learning and Soft Computing Techniques.” Energy 86: 276–91. https://doi.org/https://doi.org/10.1016/j.energy.2015.04.039.

Kansou, Kamal, and Bert Bredeweg. 2014. “Hypothesis Assessment with Qualitative Reasoning: Modelling the Fontestorbes Fountain.” Ecological Informatics 19: 71–89. https://doi.org/https://doi.org/10.1016/j.ecoinf.2013.10.007.

Karunasena, H C P, Y T Gu, R J Brown, and W Senadeera. 2015. “Numerical Investigation of Plant Tissue Porosity and Its Influence on Cellular Level Shrinkage during Drying.” Biosystems Engineering 132: 71–87. https://doi.org/https://doi.org/10.1016/j.biosystemseng.2015.02.002.

Kay, Herbert, Bernhard Rinner, and Benjamin Kuipers. 2000. “Semi-Quantitative System Identification.” Artificial Intelligence 119 (1): 103–40. https://doi.org/https://doi.org/10.1016/S0004-3702(00)00012-6.

Kelsingazina, Ruziya, Vladimir Vityuk, Alexander Vurim, Galina Vityuk, Nurzhan Mukhamedov, and Georgy Tikhomirov. 2024. “Computational Approaches for Determining the Nuclear Heating Value of Structural Materials during the Irradiation at the IGR Reactor.” Annals of Nuclear Energy 204: 110532. https://doi.org/https://doi.org/10.1016/j.anucene.2024.110532.

Khan, Faiz M, Ulf Schmitz, Svetoslav Nikolov, David Engelmann, Brigitte M Pützer, Olaf Wolkenhauer, and Julio Vera. 2014. “Hybrid Modeling of the Crosstalk between Signaling and Transcriptional Networks Using Ordinary Differential Equations and Multi-Valued Logic.” Biochimica et Biophysica Acta (BBA) - Proteins and Proteomics 1844 (1, Part B): 289–98. https://doi.org/https://doi.org/10.1016/j.bbapap.2013.05.007.

Ko, Yujeong, Hyungjoo Seo, Hyoung Kyu Cho, and Kunwoo Yi. 2024. “Effect of Inclination on SB LOCA Transient of a Floating Nuclear Reactor in MARS-KS Analysis Using Moving Reactor Model.” Nuclear Engineering and Design 426: 113407. https://doi.org/https://doi.org/10.1016/j.nucengdes.2024.113407.

Koning, Kees de, Bert Bredeweg, Joost Breuker, and Bob Wielinga. 2000. “Model-Based Reasoning about Learner Behaviour.” Artificial Intelligence 117 (2): 173–229. https://doi.org/https://doi.org/10.1016/S0004-3702(99)00106-X.

Kwamura, Koji, Yuji Naka, Tetsuo Fuchino, Atsushi Aoyama, and Nobuo Takagi. 2008. “Hazop Support System and Its Use for Operation.” In 18th European Symposium on Computer Aided Process Engineering, edited by Bertrand Braunschweig and Xavier Joulia, 25:1003–8. Computer Aided Chemical Engineering. Elsevier. https://doi.org/https://doi.org/10.1016/S1570-7946(08)80173-3.

Lahiouel, Ons, Henda Aridhi, Mohamed H Zaki, and Sofiène Tahar. 2017. “Exploiting Bounds Optimization for the Semi-Formal Verification of Analog Circuits.” Integration 59: 135–47. https://doi.org/https://doi.org/10.1016/j.vlsi.2017.06.008.

Largouët, Christine, Marie-Odile Cordier, Yves-Marie Bozec, Yulong Zhao, and Guy Fontenelle. 2012. “Use of Timed Automata and Model-Checking to Explore Scenarios on Ecosystem Models.” Environmental Modelling \& Software 30: 123–38. https://doi.org/https://doi.org/10.1016/j.envsoft.2011.08.005.

Lefeber, Erjen, Marcus Greiff, and Anders Robertsson. 2023. “A Robust Observer with Gyroscopic Bias Correction for Rotational Dynamics.” IFAC-PapersOnLine 56 (2): 1641–48. https://doi.org/https://doi.org/10.1016/j.ifacol.2023.10.1867.

Leifheit, Jana, and Rudibert King. 2005. “SYSTEMATIC STRUCTURE AND PARAMETER IDENTIFICATION FOR BIOLOGICAL REACTION SYSTEMS SUPPORTED BY A SOFTWARE-TOOL.” IFAC Proceedings Volumes 38 (1): 1095–1100. https://doi.org/https://doi.org/10.3182/20050703-6-CZ-1902.00184.

Li, Fang, Shaokuan Chen, Xiudan Wang, and Fu Feng. 2014. “Pedestrian Evacuation Modeling and Simulation on Metro Platforms Considering Panic Impacts.” Procedia - Social and Behavioral Sciences 138: 314–22. https://doi.org/https://doi.org/10.1016/j.sbspro.2014.07.209.

Li, Tiankun, Hao Xu, and Fulin Shang. 2023. “A Refined Numerical Simulation Approach to Assess the Neutron Irradiation Effect on the Mechanical Behavior of Wurtzite GaN.” Computational Materials Science 230: 112520. https://doi.org/https://doi.org/10.1016/j.commatsci.2023.112520.

Li, Weidong, How Wei Benjamin Teo, Kaijuan Chen, Jun Zeng, Kun Zhou, and Hejun Du. 2023. “Mesoscale Simulations of Spherulite Growth during Isothermal Crystallization of Polymer Melts via an Enhanced 3D Phase-Field Model.” Applied Mathematics and Computation 446: 127873. https://doi.org/https://doi.org/10.1016/j.amc.2023.127873.

Li, Yiqun, Jiahui Gao, Wei Chen, Yu Zhou, and Zhouping Yin. 2024. “Simulation and Trajectory Optimization of Articulated Robots via Spectral Variational Integrators.” Communications in Nonlinear Science and Numerical Simulation 131: 107877. https://doi.org/https://doi.org/10.1016/j.cnsns.2024.107877.

LI, Zhiheng, Dong SUN, Xuexiang JIN, Di YU, and Zuo ZHANG. 2008. “Pattern-Based Study on Urban Transportation System State Classification and Properties.” Journal of Transportation Systems Engineering and Information Technology 8 (5): 83–87. https://doi.org/https://doi.org/10.1016/S1570-6672(08)60041-0.

Liu, Isiah Po-Chun, Peter P.-Y. Chen, and Sunney I Chan. 2012. “Models for the Trinuclear Copper(II) Cluster in the Particulate Methane Monooxygenase from Methanotrophic Bacteria: Synthesis, Spectroscopic and Theoretical Characterization of Trinuclear Copper(II) Complexes.” Comptes Rendus Chimie 15 (2): 214–24. https://doi.org/https://doi.org/10.1016/j.crci.2011.11.014.

Lo, C H, K M Chow, Y K Wong, and A B Rad. 2001. “Qualitative System Identification with the Use of On-Line Genetic Algorithms.” Simulation Practice and Theory 8 (6): 415–31. https://doi.org/https://doi.org/10.1016/S0928-4869(01)00026-X.

Lu, Yunsong, Fuli Wang, Mingxing Jia, and Yuanchen Qi. 2016a. “Centrifugal Compressor Fault Diagnosis Based on Qualitative Simulation and Thermal Parameters.” Mechanical Systems and Signal Processing 81: 259–73. https://doi.org/https://doi.org/10.1016/j.ymssp.2016.03.018.

———. 2016b. “Centrifugal Compressor Fault Diagnosis Based on Qualitative Simulation and Thermal Parameters.” Mechanical Systems and Signal Processing 81 (December): 259–73. https://doi.org/10.1016/J.YMSSP.2016.03.018.

Lunze, Jan. 2000. “Process Supervision by Means of Qualitative Models.” Annual Reviews in Control 24: 41–54. https://doi.org/https://doi.org/10.1016/S1367-5788(00)90011-7.

Luo, Haixuan, Hang Zou, Baohui Liu, and Xiaoming Yue. 2020. “Fabrication of Micro Rotary Structures by Wire Electrochemical Grinding.” Electrochemistry Communications 116: 106745. https://doi.org/https://doi.org/10.1016/j.elecom.2020.106745.

Ma, Xiaomeng, and Bin Hu. 2023. “The Domination Effect of the Intelligent Environment in the Catastrophe Mechanism of Investor Behavior.” Information Processing \& Management 60 (5): 103448. https://doi.org/https://doi.org/10.1016/j.ipm.2023.103448.

Maestri, Mauricio, Daniel Ziella, Miryan Cassanello, and Gabriel Horowitz. 2014. “Automatic Qualitative Trend Simulation Method for Diagnosing Faults in Industrial Processes.” Computers \& Chemical Engineering 64: 55–62. https://doi.org/https://doi.org/10.1016/j.compchemeng.2014.01.007.

Mahboobi, S H, A Meghdari, N Jalili, and F Amiri. 2009a. “Precise Positioning and Assembly of Metallic Nanoclusters as Building Blocks of Nanostructures: A Molecular Dynamics Study.” Physica E: Low-Dimensional Systems and Nanostructures 42 (2): 182–95. https://doi.org/https://doi.org/10.1016/j.physe.2009.10.008.

———. 2009b. “Qualitative Study of Nanocluster Positioning Process: Planar Molecular Dynamics Simulations.” Current Applied Physics 9 (5): 997–1004. https://doi.org/https://doi.org/10.1016/j.cap.2008.10.006.

Maurya, Mano Ram, Raghunathan Rengaswamy, and Venkat Venkatasubramanian. 2006. “A Signed Directed Graph-Based Systematic Framework for Steady-State Malfunction Diagnosis inside Control Loops.” Chemical Engineering Science 61 (6): 1790–1810. https://doi.org/https://doi.org/10.1016/j.ces.2005.10.023.

———. 2007. “Fault Diagnosis Using Dynamic Trend Analysis: A Review and Recent Developments.” Engineering Applications of Artificial Intelligence 20 (2): 133–46. https://doi.org/https://doi.org/10.1016/j.engappai.2006.06.020.

Montaño-Moctezuma, Gabriela, Hiram W Li, and Philippe A Rossignol. 2007. “Alternative Community Structures in a Kelp-Urchin Community: A Qualitative Modeling Approach.” Ecological Modelling 205 (3): 343–54. https://doi.org/https://doi.org/10.1016/j.ecolmodel.2007.02.031.

———. 2008. “Variability of Community Interaction Networks in Marine Reserves and Adjacent Exploited Areas.” Fisheries Research 94 (1): 99–108. https://doi.org/https://doi.org/10.1016/j.fishres.2008.07.003.

Monteiro, P T, W Abou-Jaoudé, D Thieffry, and C Chaouiya. 2014. “Model Checking Logical Regulatory Networks.” IFAC Proceedings Volumes 47 (2): 170–75. https://doi.org/https://doi.org/10.3182/20140514-3-FR-4046.00135.

Mousavizadeh, Seyed Reza, Ramtin Moeini, and Ahmad Shanehsazzadeh. 2023. “Management of Aquifer and Dam Reservoir Quantitative-Qualitative Interaction.” Agricultural Water Management 277: 108116. https://doi.org/https://doi.org/10.1016/j.agwat.2022.108116.

Müller-Eping, Thorsten, and Gerwald Lichtenberg. 2020. “Simulation with Qualitative Models in Reduced Tensor Representations.” IFAC-PapersOnLine 53 (2): 2108–15. https://doi.org/https://doi.org/10.1016/j.ifacol.2020.12.2532.

Muzy, A, J J Nutaro, B P Zeigler, and P Coquillard. 2008. “Modeling and Simulation of Fire Spreading through the Activity Tracking Paradigm.” Ecological Modelling 219 (1): 212–25. https://doi.org/https://doi.org/10.1016/j.ecolmodel.2008.08.017.

Neugebohrn, Nils, Maria S Hammer, Janet Neerken, Jürgen Parisi, and Ingo Riedel. 2015. “Analysis of the Back Contact Properties of Cu(In,Ga)Se2 Solar Cells Employing the Thermionic Emission Model.” Thin Solid Films 582: 332–35. https://doi.org/https://doi.org/10.1016/j.tsf.2014.10.073.

Okamoto, Hiromi, Kohei Imura, Toru Shimada, and Masahiro Kitajima. 2011. “Spatial Distribution of Enhanced Optical Fields in Monolayered Assemblies of Metal Nanoparticles: Effects of Interparticle Coupling.” Journal of Photochemistry and Photobiology A: Chemistry 221 (2): 154–59. https://doi.org/https://doi.org/10.1016/j.jphotochem.2011.01.017.

Ortiz, Marco, and Liliana Ayala. 2024. “Assessing the Use of Fisher Knowledge in Oceanic Eco-Social Networks of Peru: A Comprehensive Approach to Sustainable Fisheries Management and Governance.” Ocean \& Coastal Management 255: 107217. https://doi.org/https://doi.org/10.1016/j.ocecoaman.2024.107217.

Palmer, C, and P W H Chung. 2009. “An Automated System for Batch Hazard and Operability Studies.” Reliability Engineering \& System Safety 94 (6): 1095–1106. https://doi.org/https://doi.org/10.1016/j.ress.2009.01.001.

Pang, Wei, and George M Coghill. 2015. “Qualitative, Semi-Quantitative, and Quantitative Simulation of the Osmoregulation System in Yeast.” Biosystems 131: 40–50. https://doi.org/https://doi.org/10.1016/j.biosystems.2015.04.003.

Papastavrou, Areti, Ina Schmidt, Kefu Deng, and Paul Steinmann. 2020. “On Age-Dependent Bone Remodeling.” Journal of Biomechanics 103: 109701. https://doi.org/https://doi.org/10.1016/j.jbiomech.2020.109701.

Paul Frank, Μ, X Steven Ding, and Birgit Koppen-seliger. 2000. “Current Developments in the Theory of FDI.” IFAC Proceedings Volumes 33 (11): 17–28. https://doi.org/https://doi.org/10.1016/S1474-6670(17)37336-6.

Pausas, Juli G, and Juan I Ramos. 2006. “Landscape Analysis and Simulation Shell (Lass).” Environmental Modelling \& Software 21 (5): 629–39. https://doi.org/https://doi.org/10.1016/j.envsoft.2004.11.009.

Payan, Yohan, Matthieu Chabanas, Xavier Pelorson, Coriandre Vilain, Patrick Levy, Vincent Luboz, and Pascal Perrier. 2002. “Biomechanical Models to Simulate Consequences of Maxillofacial Surgery.” Comptes Rendus Biologies 325 (4): 407–17. https://doi.org/https://doi.org/10.1016/S1631-0691(02)01443-9.

Plant, Richard E, and Marc P Vayssières. 2000. “Combining Expert System and GIS Technology to Implement a State-Transition Model of Oak Woodlands.” Computers and Electronics in Agriculture 27 (1): 71–93. https://doi.org/https://doi.org/10.1016/S0168-1699(00)00099-5.

Porreca, Riccardo, Samuel Drulhe, Hidde de Jong, and Giancarlo Ferrari-Trecate. 2009. “Identification of Parameters and Structure of Piecewise Affine Models of Genetic Networks.” IFAC Proceedings Volumes 42 (10): 587–92. https://doi.org/https://doi.org/10.3182/20090706-3-FR-2004.00097.

Postigo-Boix, Marcos, Joan Garcia-Haro, and Jose L Melús-Moreno. 2007. “A Cost-Efficient Method for Streaming Stored Content in a Guaranteed QoS Internet.” Computer Networks 51 (1): 309–35. https://doi.org/https://doi.org/10.1016/j.comnet.2006.05.002.

Potočnik, Jaka, Luka Pajek, and Mitja Košir. 2024. “Experimental Investigation of the Impact of Model Complexity on Indoor Daylight Spectral Simulations.” Developments in the Built Environment 20: 100543. https://doi.org/https://doi.org/10.1016/j.dibe.2024.100543.

Price, C J, and N A Snooke. 2006. “IMPLEMENTING A LAYERED APPROACH TO AUTOMATED SAFETY ANALYSIS.” IFAC Proceedings Volumes 39 (13): 1151–56. https://doi.org/https://doi.org/10.3182/20060829-4-CN-2909.00192.

Price, C J, and N S Taylor. 2002. “Automated Multiple Failure FMEA.” Reliability Engineering \& System Safety 76 (1): 1–10. https://doi.org/https://doi.org/10.1016/S0951-8320(01)00136-3.

Qu, Yuanwei, Eduard Kamburjan, Anita Torabi, and Martin Giese. 2024. “Semantically Triggered Qualitative Simulation of a Geological Process.” Applied Computing and Geosciences 21: 100152. https://doi.org/https://doi.org/10.1016/j.acags.2023.100152.

Rahimi, Mohammad J, Hariswaran Sitaraman, David Humbird, and Jonathan J Stickel. 2018. “Computational Fluid Dynamics Study of Full-Scale Aerobic Bioreactors: Evaluation of Gas–Liquid Mass Transfer, Oxygen Uptake, and Dynamic Oxygen Distribution.” Chemical Engineering Research and Design 139: 283–95. https://doi.org/https://doi.org/10.1016/j.cherd.2018.08.033.

Rahnama, Mohammad Bagher, and Abbas Zamzam. 2013. “Quantitative and Qualitative Simulation of Groundwater by Mathematical Models in Rafsanjan Aquifer Using MODFLOW and MT3DMS.” Arabian Journal of Geosciences 6 (3): 901–12. https://doi.org/10.1007/S12517-011-0364-X.

Rajasekharan, Ajaykumar, and Charbel Farhat. 2009. “Applications of a Variational Multiscale Method for Large Eddy Simulation of Turbulent Flows on Moving/Deforming Unstructured Grids.” Finite Elements in Analysis and Design 45 (4): 272–79. https://doi.org/https://doi.org/10.1016/j.finel.2008.10.013.

Ram Maurya, Mano, Raghunathan Rengaswamy, and Venkat Venkatasubramanian. 2004. “Application of Signed Digraphs-Based Analysis for Fault Diagnosis of Chemical Process Flowsheets.” Engineering Applications of Artificial Intelligence 17 (5): 501–18. https://doi.org/https://doi.org/10.1016/j.engappai.2004.03.007.

Rebolledo, M. 2006. “Rough Intervals—Enhancing Intervals for Qualitative Modeling of Technical Systems.” Artificial Intelligence 170 (8): 667–85. https://doi.org/https://doi.org/10.1016/j.artint.2006.02.004.

Reil, Matt, Joseph Hoffman, Paul Predecki, and Maciej Kumosa. 2022. “Graphene and Graphene Oxide Energetic Interactions with Polymers through Molecular Dynamics Simulations.” Computational Materials Science 211: 111548. https://doi.org/https://doi.org/10.1016/j.commatsci.2022.111548.

Ren, Jian-kun, Ming-yue Sun, Yun Chen, Bin Xu, Wei-feng Liu, Hai-yang Jiang, Yan-fei Cao, and Dian-zhong Li. 2022. “The Non-Dendritic Microstructure Arising from Grain Boundary Formation and Wetting: A Phase-Field Simulation and Experimental Investigation of Semi-Solid Deformation.” Materials \& Design 223: 111111. https://doi.org/https://doi.org/10.1016/j.matdes.2022.111111.

Ropers, Delphine, Valentina Baldazzi, and Hidde de Jong. 2009. “Reduction of a Kinetic Model of the Carbon Starvation Response in Escherichia Coli.” IFAC Proceedings Volumes 42 (10): 27–32. https://doi.org/https://doi.org/10.3182/20090706-3-FR-2004.00003.

Ropers, Delphine, Hidde de Jong, Michel Page, Dominique Schneider, and Johannes Geiselmann. 2006. “Qualitative Simulation of the Carbon Starvation Response in Escherichia Coli.” Biosystems 84 (2): 124–52. https://doi.org/10.1016/J.BIOSYSTEMS.2005.10.005.

Rubtsov, Nickolai M, Victor I Chernysh, Georgii I Tsvetkov, and Kirill Ya. Troshin. 2021. “Interaction of the Combustion Front of Methane-Air Mixture at Low Pressures with Obstacles of Cylindrical Shape.” FirePhysChem 1 (3): 174–78. https://doi.org/https://doi.org/10.1016/j.fpc.2021.07.001.

Rubtsov, Nikolai M, Victor I Chernysh, Georgii I Tsvetkov, and Kirill Ya. Troshin. 2021. “Interaction between Laminar Flames of Natural Gas–Oxygen Mixtures and Planar Obstacles with Asymmetrical Openings.” Mendeleev Communications 31 (1): 132–34. https://doi.org/https://doi.org/10.1016/j.mencom.2021.01.043.

Sacci, Robert L, Lance W Gill, Edward W Hagaman, and Nancy J Dudney. 2015. “Operando NMR and XRD Study of Chemically Synthesized LiCx Oxidation in a Dry Room Environment.” Journal of Power Sources 287: 253–60. https://doi.org/https://doi.org/10.1016/j.jpowsour.2015.04.035.

Salles, Paulo, Bert Bredeweg, and Symone Araújo. 2006. “Qualitative Models about Stream Ecosystem Recovery: Exploratory Studies.” Ecological Modelling 194 (1): 80–89. https://doi.org/https://doi.org/10.1016/j.ecolmodel.2005.10.018.

Savkovic-Stevanovic, Jelenka. 2007. “Process Plant Risk Analysis and Modelling.” In 17th European Symposium on Computer Aided Process Engineering, edited by Valentin Pleşu and Paul Şerban Agachi, 24:1229–34. Computer Aided Chemical Engineering. Elsevier. https://doi.org/https://doi.org/10.1016/S1570-7946(07)80229-X.

Schaich, David, Ralf Becker, and Rudibert King. 2000. “Qualitative Modelling as a Key Technique for the Automatic Identification of Mathematical Models of Chemical Reaction Systems.” IFAC Proceedings Volumes 33 (15): 421–26. https://doi.org/https://doi.org/10.1016/S1474-6670(17)39787-2.

———. 2001. “Qualitative Modelling for Automatic Identification of Mathematical Models of Chemical Reaction Systems.” Control Engineering Practice 9 (12): 1373–81. https://doi.org/https://doi.org/10.1016/S0967-0661(01)00080-6.

Seiz, Marco, Michael Kellner, and Britta Nestler. 2023. “Simulation of Dendritic–Eutectic Growth with the Phase-Field Method.” Acta Materialia 254: 118965. https://doi.org/https://doi.org/10.1016/j.actamat.2023.118965.

Sepulchre, Jacques-A., Sylvie Reverchon, and William Nasser. 2007. “Modeling the Onset of Virulence in a Pectinolytic Bacterium.” Journal of Theoretical Biology 244 (2): 239–57. https://doi.org/https://doi.org/10.1016/j.jtbi.2006.08.010.

SHEN, Li-juan, Yan-feng HU, Jian-zhong CHEN, Peng ZHANG, and Hua-zhen DAI. 2009. “Numerical Simulation of the Flow Field in a Dense-Media Cyclone.” Mining Science and Technology (China) 19 (2): 225–29. https://doi.org/https://doi.org/10.1016/S1674-5264(09)60043-0.

Söffker, Dir. 2000. “Human-Machine Interaction: Modeling of Individual Planning, Cognition, Representation and Action.” IFAC Proceedings Volumes 33 (12): 129–32. https://doi.org/https://doi.org/10.1016/S1474-6670(17)37291-9.

Song, Guofeng, Ming Shao, Lunlin Shang, Yi Zhou, You Lv, Xu Wang, Jianbei Liu, and Zhiyong Zhang. 2020. “Production and Properties of a Charging-up ‘Free’ THGEM with DLC Coating.” Nuclear Instruments and Methods in Physics Research Section A: Accelerators, Spectrometers, Detectors and Associated Equipment 966: 163868. https://doi.org/https://doi.org/10.1016/j.nima.2020.163868.

Song, Hao, Paul Smolen, Evyatar Av-Ron, Douglas A Baxter, and John H Byrne. 2007. “Dynamics of a Minimal Model of Interlocked Positive and Negative Feedback Loops of Transcriptional Regulation by CAMP-Response Element Binding Proteins.” Biophysical Journal 92 (10): 3407–24. https://doi.org/https://doi.org/10.1529/biophysj.106.096891.

Soni, Kaushal, and Arvind Kumar Sinha. 2024. “Modeling and Stability Analysis of the Transmission Dynamics of Monkeypox with Control Intervention.” Partial Differential Equations in Applied Mathematics 10: 100730. https://doi.org/https://doi.org/10.1016/j.padiff.2024.100730.

Sun, Jianmeng, Jianshen Gao, Yanjiao Jiang, and Likai Cui. 2016. “Resistivity and Relative Permittivity Imaging for Oil-Based Mud: A Method and Numerical Simulation.” Journal of Petroleum Science and Engineering 147: 24–33. https://doi.org/https://doi.org/10.1016/j.petrol.2016.04.042.

Suzuki, Yuki, Kunikazu Suzuki, Masaki Michihata, Kiyoshi Takamasu, and Satoru Takahashi. 2018. “One-Shot Stereolithography for Biomimetic Micro Hemisphere Covered with Relief Structure.” Precision Engineering 54: 353–60. https://doi.org/https://doi.org/10.1016/j.precisioneng.2018.07.004.

Tabasy, Mohammad, and Fariborz Rashidi. 2015. “A Qualitative Simulation of a Face Dissolution Pattern in Acidizing Process Using Rotating Disk Apparatus for a Carbonate Gas Reservoir.” Journal of Natural Gas Science and Engineering 26: 1460–69. https://doi.org/https://doi.org/10.1016/j.jngse.2015.08.014.

Taghizadeh, Soudabeh, Salar Khani, and Taher Rajaee. 2021. “Hybrid SWMM and Particle Swarm Optimization Model for Urban Runoff Water Quality Control by Using Green Infrastructures (LID-BMPs).” Urban Forestry \& Urban Greening 60: 127032. https://doi.org/https://doi.org/10.1016/j.ufug.2021.127032.

Tarifa, Enrique E, and Nicolás J Scenna. 2003. “Fault Diagnosis for a MSF Using a SDG and Fuzzy Logic.” Desalination 152 (1): 207–14. https://doi.org/https://doi.org/10.1016/S0011-9164(02)01065-2.

———. 2004. “Fault Diagnosis for MSF Dynamic States Using a SDG and Fuzzy Logic.” Desalination 166: 93–101. https://doi.org/https://doi.org/10.1016/j.desal.2004.06.063.

Tikka, Pauli, Moritz Mercker, Ilya Skovorodkin, Ulla Saarela, Seppo Vainio, Veli-Pekka Ronkainen, James P Sluka, James A Glazier, Anna Marciniak-Czochra, and Franz Schaefer. 2022. “Computational Modelling of Nephron Progenitor Cell Movement and Aggregation during Kidney Organogenesis.” Mathematical Biosciences 344: 108759. https://doi.org/https://doi.org/10.1016/j.mbs.2021.108759.

Timma, Lelde, and Dagnija Blumberga. 2015. “An Algorithm for the Selection of Structure for Artificial Networks. Case Study: Solar Thermal Energy Systems.” Energy Procedia 72: 135–41. https://doi.org/https://doi.org/10.1016/j.egypro.2015.06.019.

Tsai, Jerry Jen-Hung, and John S Gero. 2010. “A Qualitative Energy-Based Unified Representation for Buildings.” Automation in Construction 19 (1): 20–42. https://doi.org/https://doi.org/10.1016/j.autcon.2009.07.007.

Tullos, Desiree D., and Michael Neumann. 2006. “A Qualitative Model for Analyzing the Effects of Anthropogenic Activities in the Watershed on Benthic Macroinvertebrate Communities.” Ecological Modelling 196 (1–2): 209–20. https://doi.org/10.1016/J.ECOLMODEL.2006.02.018.

Ursino, Mauro, Silvana Pelle, Fahima Nekka, Philippe Robaey, and Miriam Schirru. 2024. “Valence-Dependent Dopaminergic Modulation during Reversal Learning in Parkinson’s Disease: A Neurocomputational Approach.” Neurobiology of Learning and Memory 215: 107985. https://doi.org/https://doi.org/10.1016/j.nlm.2024.107985.

Vakhrushev, Andrey v, and Alexei A Gorbunov. 2016. “Theory of Chromatography of Partially Cyclic Polymers: Tadpole-Type and Manacle-Type Macromolecules.” Journal of Chromatography A 1433: 56–65. https://doi.org/https://doi.org/10.1016/j.chroma.2015.12.042.

Venkatasubramanian, Venkat, Raghunathan Rengaswamy, and Surya N Kavuri. 2003. “A Review of Process Fault Detection and Diagnosis: Part II: Qualitative Models and Search Strategies.” Computers \& Chemical Engineering 27 (3): 313–26. https://doi.org/https://doi.org/10.1016/S0098-1354(02)00161-8.

Vobruba, Tamara, Andreas Körner, and Felix Breitenecker. 2016. “Modelling, Analysis and Simulation of a Spatial Interaction Model.” IFAC-PapersOnLine 49 (29): 221–25. https://doi.org/https://doi.org/10.1016/j.ifacol.2016.11.054.

Vrancken, Mark, Yves Schols, Wim Aerts, and Guy A E Vandenbosch. 2007. “Benchmark of Full Maxwell 3-Dimensional Electromagnetic Field Solvers on Prototype Cavity-Backed Aperture Antenna.” AEU - International Journal of Electronics and Communications 61 (6): 363–69. https://doi.org/https://doi.org/10.1016/j.aeue.2006.07.001.

Wang, Haoming, Haibo Zhao, Zhaoli Guo, and Chuguang Zheng. 2012. “Numerical Simulation of Particle Capture Process of Fibrous Filters Using Lattice Boltzmann Two-Phase Flow Model.” Powder Technology 227: 111–22. https://doi.org/https://doi.org/10.1016/j.powtec.2011.12.057.

Wang, Li, Ping Wang, Rong Chen, Shaoguang Li, and Zili Li. 2020. “Experimental and Numerical Investigation of Damage Development in Embedded Rail System under Longitudinal Force.” Engineering Failure Analysis 114: 104590. https://doi.org/https://doi.org/10.1016/j.engfailanal.2020.104590.

Wang, Liyu, Jack Hodges, Dan Yu, and Ronald S Fearing. 2021. “Automatic Modeling and Fault Diagnosis of Car Production Lines Based on First-Principle Qualitative Mechanics and Semantic Web Technology.” Advanced Engineering Informatics 49: 101248. https://doi.org/https://doi.org/10.1016/j.aei.2021.101248.

Wang, Qichun, Mingxiang Zhang, and Sama Abdolhosseinzadeh. 2024. “Application of Modified Seagull Optimization Algorithm with Archives in Urban Water Distribution Networks: Dealing with the Consequences of Sudden Pollution Load.” 
Heliyon 10 (3): e24920. https://doi.org/https://doi.org/10.1016/j.heliyon.2024.e24920.

Wedel, Dipl.-Ing. M, and H C P Göhner. 2006. “HOLISTIC QUALITATIVE AND MODEL-BASED RELIABILITY ANALYSIS OF PROGRAMMABLE MECHATRONIC SYSTEMS.” IFAC Proceedings Volumes 39 (16): 91–96. https://doi.org/https://doi.org/10.3182/20060912-3-DE-2911.00019.

Wohlfahrt, Georg, and Marta Galvagno. 2017. “Revisiting the Choice of the Driving Temperature for Eddy Covariance CO2 Flux Partitioning.” Agricultural and Forest Meteorology 237–238: 135–42. https://doi.org/https://doi.org/10.1016/j.agrformet.2017.02.012.

Wood, I, M F Martini, J M R Albano, M L Cuestas, V L Mathet, and M Pickholz. 2016. “Coarse Grained Study of Pluronic F127: Comparison with Shorter Co-Polymers in Its Interaction with Lipid Bilayers and Self-Aggregation in Water.” Journal of Molecular Structure 1109: 106–13. https://doi.org/https://doi.org/10.1016/j.molstruc.2015.12.073.

Wu, Jiang, Bin Hu, Jinlong Zhang, and Da Fang. 2008. “Multi-Agent Simulation of Group Behavior in E-Government Policy Decision.” Simulation Modelling Practice and Theory 16 (10): 1571–87. https://doi.org/https://doi.org/10.1016/j.simpat.2007.07.007.

Xu, Run-Ze, Jia-Shun Cao, Ganyu Feng, Jing-Yang Luo, Yang Wu, Bing-Jie Ni, and Fang Fang. 2021. “Modeling Molecular Structure and Behavior of Microbial Extracellular Polymeric Substances through Interacting-Particle Reaction Dynamics.” Chemical Engineering Journal Advances 8: 100154. https://doi.org/https://doi.org/10.1016/j.ceja.2021.100154.

Yaghoubi, Behrouz, Seyed Abbas Hosseini, Sara Nazif, and Amin Daghighi. 2020. “Development of Reservoir’s Optimum Operation Rules Considering Water Quality Issues and Climatic Change Data Analysis.” Sustainable Cities and Society 63: 102467. https://doi.org/https://doi.org/10.1016/j.scs.2020.102467.

Yan, Liu, Chen Wei, Yang Shanchao, Jin Xiaoming, and He Chaohui. 2016. “Synergistic Effect of Mixed Neutron and Gamma Irradiation in Bipolar Operational Amplifier OP07.” Nuclear Instruments and Methods in Physics Research Section A: Accelerators, Spectrometers, Detectors and Associated Equipment 831: 334–38. https://doi.org/https://doi.org/10.1016/j.nima.2016.05.072.

Yang, Bojie, Zhuoqin Yang, Heng Liu, and Hong Qi. 2023. “Dynamic Modelling and Tristability Analysis of Misfolded α-Synuclein Degraded via Autophagy in Parkinson’s Disease.” Biosystems 233: 105036. https://doi.org/https://doi.org/10.1016/j.biosystems.2023.105036.

Yang, Fan, Sirish L Shah, and Deyun Xiao. 2010. “SDG (Signed Directed Graph) Based Process Description and Fault Propagation Analysis for a Tailings Pumping Process.” IFAC Proceedings Volumes 43 (9): 50–55. https://doi.org/https://doi.org/10.3182/20100802-3-ZA-2014.00011.

Yang, Jinsong, Jingsong Xie, Tiantian Wang, Jinglong Chen, and Yanyang Zi. 2022. “An Identification Method for Excitation Location and Its Application in Faults Location.” Mechanical Systems and Signal Processing 171: 108948. https://doi.org/https://doi.org/10.1016/j.ymssp.2022.108948.

Yang, Konghua, Qi Liu, Zhaohua Lin, Yunhong Liang, and Chunbao Liu. 2022. “Investigations of Interfacial Heat Transfer and Phase Change on Bioinspired Superhydrophobic Surface for Anti-Icing/de-Icing.” International Communications in Heat and Mass Transfer 134: 105994. https://doi.org/https://doi.org/10.1016/j.icheatmasstransfer.2022.105994.

Yu, Kai, Qinggui Cao, and Lujie Zhou. 2019. “Study on Qualitative Simulation Technology of Group Safety Behaviors and the Related Software Platform.” Computers \& Industrial Engineering 127: 1037–55. https://doi.org/https://doi.org/10.1016/j.cie.2018.11.040.

Zhang, Ming-jun, Yu-jia Wang, Jian-an Xu, and Zheng-chen Liu. 2015. “Thruster Fault Diagnosis in Autonomous Underwater Vehicle Based on Grey Qualitative Simulation.” Ocean Engineering 105: 247–55. https://doi.org/https://doi.org/10.1016/j.oceaneng.2015.06.037.

ZHANG, Weihua, Chongguang WU, and Chunli WANG. 2011. “Qualitative Algebra and Graph Theory Methods for Dynamic Trend Analysis of Continuous System.” Chinese Journal of Chemical Engineering 19 (2): 308–15. https://doi.org/https://doi.org/10.1016/S1004-9541(11)60170-5.

ZHANG, Yuliang, Beike ZHANG, Xin MA, Liulin CAO, and Chongguang WU. 2013. “Consequence Identification for Maloperation in Batch Process.” Chinese Journal of Chemical Engineering 21 (12): 1347–59. https://doi.org/https://doi.org/10.1016/S1004-9541(13)60620-5.

Zhang, Yuliang, Wentao Zhang, and Beike Zhang. 2015. “Automatic HAZOP Analysis Method for Unsteady Operation in Chemical Based on Qualitative Simulation and Inference.” Chinese Journal of Chemical Engineering 23 (12): 2065–74. https://doi.org/https://doi.org/10.1016/j.cjche.2015.10.004.

Zhang, Zhao-qian, Chong-guang Wu, Bei-ke Zhang, Tao Xia, and An-feng Li. 2005. “SDG Multiple Fault Diagnosis by Real-Time Inverse Inference.” Reliability Engineering \& System Safety 87 (2): 173–89. https://doi.org/https://doi.org/10.1016/j.ress.2004.04.008.

Zhan-yin, Ye, Wei Feng-si, Wang Chi, Feng Xue-shan, and Xiang Chang-ging. 2004. “An Improvement on the Numerical Simulation of CME Event between Two Coronal Streamers.” Chinese Astronomy and Astrophysics 28 (3): 323–30. https://doi.org/https://doi.org/10.1016/j.chinastron.2004.07.007.

Zhao, Yufan, Kenta Aoyagi, Kenta Yamanaka, and Akihiko Chiba. 2020. “Role of Operating and Environmental Conditions in Determining Molten Pool Dynamics during Electron Beam Melting and Selective Laser Melting.” Additive Manufacturing 36: 101559. https://doi.org/https://doi.org/10.1016/j.addma.2020.101559.

Zheng, Yihua, Chengchun Zhang, Jing Wang, Yan Liu, Chun Shen, and Junfeng Yang. 2019. “Robust Adhesion of Droplets via Heterogeneous Dynamic Petal Effects.” Journal of Colloid and Interface Science 557: 737–45. https://doi.org/https://doi.org/10.1016/j.jcis.2019.09.070.

Zhong, Chen, S Thomas Ng, and Martin Skitmore. 2021. “An Interdependent Infrastructure Asset Management Framework for High-Density Cities.” Proceedings of the Institution of Civil Engineers - Municipal Engineer 174 (3): 180–90. https://doi.org/https://doi.org/10.1680/jmuen.18.00053.

Zolfagharipoor, Mohammad Amin, and Azadeh Ahmadi. 2016. “A Decision-Making Framework for River Water Quality Management under Uncertainty: Application of Social Choice Rules.” Journal of Environmental Management 183: 152–63. https://doi.org/https://doi.org/10.1016/j.jenvman.2016.07.094.


\end{document}
